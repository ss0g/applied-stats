\documentclass[12pt, letterpaper]{article}
\usepackage[letterpaper, portrait, margin=1in]{geometry}

\title{Honors Applied Statistics Final Project Proposal}
\author{Troy Edwards}
\date{2023--04--14}

\begin{document}

\maketitle

\section{Research question}

The question that I seek to answer with my research is: ``What factors influence an FRC team's success?'' I chose this
question because I myself am on an FRC team and I am interested to see what factors influence success to try to improve
my own team's success.

\section{Related literature}

There is not much work that has already been done on this topic. I managed to find 2 papers: ``An Analysis of the
Success of FRC Robotics Teams'' by Max Tepermeister and ``An Overview and Analysis of Statistics used to Rate FIRST
Robotics Teams'' by William Gardner. The first one is not a paper that I would consider ``scholarly'' because it is a
high-schooler's final project for AP Statistics, but the second one appears formal and well-organized. However,
Gardner's paper is not fully on-topic, as it analyzes metrics used to rate FRC teams. The information presented is
still relevant, just not directly applicable. Tepermeister's objective was similar to mine. Tepermeister mainly
analyzed the effects of the age, size, and budget of a team on its success. Tepermeister found that none of these
factors have a significant effect on success. 

\section{Gaps in current research}

Tepermeister's research only analyzed a few factors, and was more of a general analysis and did not address performance
in specific games, where robot design can have a huge influence on performance. I seek to build on Tepermeister's
research with similar analysis, along with new research analyzing teams' capabilities in robot design in the recent FRC
games RAPID REACT (2022) and CHARGED UP (2023). Gardner's research only analyzed the effectiveness of various
techniques and metrics used to rate FRC teams and measure their performance. I can use Gardner's research to decide
what methods to use to measure success. 

\section{Data sources}

I plan to use Evan Kuykendall's 2023 scouting data (not publicly available to my knowledge) and Jake Benjamin's 2022
scouting data (also not publicly available to my knowledge). I also plan to use data from The Blue Alliance,
Statbotics, and the FRC Events API.\@ 

\section{Research methods}

I plan to take a random sample of 30 teams from the Pacific Northwest district for each year that I analyze. I will use
teams from the Pacific Northwest for a more in-depth analysis because the data that I have access to is much more
in-depth and easy to use for this region. I will, however, also do an analysis of 60 randomly selected teams globally
to see if there is any difference in what makes them successful. This will also allow me to use country as a variable
to predict success.

\section{Predictive models and inference tests}

I plan to do use the same process to analyze both RAPID REACT and CHARGED UP for the Pacific Northwest and for the
entire world.\@ This process includes 3 main steps:
\begin{enumerate}
    \item Create a multiple linear regression model to predict a team's OPR (Offensive Power Rating) for the season.
    \item Create a multiple logistic regression model to predict the probability of a specific team winning in any 
    given match.
    \item Run ANOVA tests to determine whether various robot features have a significant effect on a robot's success.
\end{enumerate}

\section{Expected challenges}

I am expecting that there will be a couple of teams in the Pacific Northwest that will not have been included in the
scouting data because they were not at the events where the data was gathered. Another challenge is that data from The
Blue Alliance and the FRC Events API is only about the entire alliance, not specific teams. Every match in every FRC
game is a 3 vs. 3, which will make it more difficult to analyze specific teams, but will also let me use data about
alliance members and opponents to predict whether a team will win a match or not. 

\end{document}